
\documentclass{article}
% \usepackage{dtk-logos} % for TeX logos
\usepackage{fontspec}

% 更纱黑体: Sarasa Gothic SC       sarasa-gothic-sc-regular.ttf 
% 思源黑体: Source Han Sans CN     ourceHanSansCN-Normal.otf 
% 微软雅黑: Microsoft YaHei        msyhl.ttc 
% \setmainfont{Sarasa Gothic SC}
% \setmainfont{Source Han Sans CN}
% \setmainfont{Microsoft YaHei}

%% Mono
% NotoSansMonoCJKsc-Regular.otf: Noto Sans Mono CJK SC
% JuliaMono-Regular.ttf: JuliaMono
% \setmainfont{Noto Sans Mono CJK SC}
% \setmainfont{JuliaMono}

%% Serif
% times.ttf: Times New Roman
% cour.ttf: Courier New
% \setmainfont{times.ttf}
\setmainfont{cour.ttf}


\begin{document}

% 字体名称
{\LARGE Courier New }

%% 测试文本出自 fontspec 的手册
% https://github.com/wspr/fontspec/blob/master/fontspec-doc-fontsel.tex#L83-L131
\subsection{By font name}
Fonts known to Lua\TeX\ or Xe\TeX\ may be loaded by their standard names as
you'd speak them out loud, such as \emph{Times New Roman} or
\emph{Adobe Garamond}.
`Known to' in this case generally means `exists in a standard fonts location'
such as \verb|~/Library/Fonts| on MacOSX, or \verb|C:\Windows\Fonts| on Windows.
In Lua\TeX, fonts found in the \textsc{texmf} tree can also be loaded by name.
In Xe\TeX, fonts found in the \textsc{texmf} tree can be loaded in Windows and Linux,
but not on MacOSX.


\subsection{By file name}
Xe\TeX\ and Lua\TeX\ also allow fonts to be loaded by file name instead of font name.
When you have a very large collection of fonts, you will sometimes not
wish to have them all installed in your system's font directories.
In this case, it is more convenient to load them from a different location on your disk.
This technique is also necessary in Xe\TeX\ when loading OpenType fonts that are present within your \TeX\ distribution, such as \verb|/usr/local/texlive/2013/texmf-dist/fonts/opentype/public|.
Fonts in such locations are visible to Xe\TeX\ but cannot be loaded by font name, only file name; Lua\TeX\ does not have this restriction.


\end{document}